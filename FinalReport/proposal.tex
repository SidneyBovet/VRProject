%%%%%%%%%%%%%%%%%%%%%%%%%%%%%%%%%%%%%%%%%
% Short Sectioned Assignment
% LaTeX Template
% Version 1.0 (5/5/12)
%
% This template has been downloaded from:
% http://www.LaTeXTemplates.com
%
% Original author:
% Frits Wenneker (http://www.howtotex.com)
%
% License:
% CC BY-NC-SA 3.0 (http://creativecommons.org/licenses/by-nc-sa/3.0/)
%
%%%%%%%%%%%%%%%%%%%%%%%%%%%%%%%%%%%%%%%%%

%--------------------------------------------------------------------------------
%	PACKAGES AND OTHER DOCUMENT CONFIGURATIONS
%--------------------------------------------------------------------------------

\documentclass[paper=a4, fontsize=11pt]{scrartcl} % A4 paper and 11pt font size

\usepackage{graphicx}
\usepackage{caption}
\usepackage{subcaption}
\usepackage[T1]{fontenc} % Use 8-bit encoding that has 256 glyphs
\usepackage{fourier} % Use the Adobe Utopia font for the document - comment this line to return to the LaTeX default
\usepackage[english]{babel} % English language/hyphenation
\usepackage{amsmath,amsfonts,amsthm} % Math packages

\usepackage{sectsty} % Allows customizing section commands
\allsectionsfont{\centering \normalfont\scshape} % Make all sections centered, the default font and small caps

\usepackage{fancyhdr} % Custom headers and footers
\pagestyle{fancyplain} % Makes all pages in the document conform to the custom headers and footers
\fancyhead{} % No page header - if you want one, create it in the same way as the footers below
\fancyfoot[L]{} % Empty left footer
\fancyfoot[C]{} % Empty center footer
\fancyfoot[R]{\thepage} % Page numbering for right footer
\renewcommand{\headrulewidth}{0pt} % Remove header underlines
\renewcommand{\footrulewidth}{0pt} % Remove footer underlines
\setlength{\headheight}{13.6pt} % Customize the height of the header

\numberwithin{equation}{section} % Number equations within sections (i.e. 1.1, 1.2, 2.1, 2.2 instead of 1, 2, 3, 4)
\numberwithin{figure}{section} % Number figures within sections (i.e. 1.1, 1.2, 2.1, 2.2 instead of 1, 2, 3, 4)
\numberwithin{table}{section} % Number tables within sections (i.e. 1.1, 1.2, 2.1, 2.2 instead of 1, 2, 3, 4)

\setlength\parindent{0pt} % Removes all indentation from paragraphs - comment this line for an assignment with lots of text

\usepackage[hidelinks]{hyperref}

%----------------------------------------------------------------------------------------
%	TITLE SECTION
%----------------------------------------------------------------------------------------

\newcommand{\horrule}[1]{\rule{\linewidth}{#1}} % Create horizontal rule command with 1 argument of height

\title{
	\normalfont \normalsize
	\textsc{EPFL CS-444 Virtual Reality} \\ [25pt] % Your university, school and/or department name(s)
	\horrule{0.5pt} \\[0.4cm] % Thin top horizontal rule
	\huge File Explorer in Outer Space \\ % The assignment title
	\horrule{2pt} \\[0.5cm] % Thick bottom horizontal rule
}

\author{Group 3: Florian Junker, Sidney Bovet} % Your name

\date{\normalsize\today} % Today's date or a custom date

\begin{document}

\maketitle % Print the title


% The report is meant to inform the structure/architecture of your project and instruct on how to configure and run it.


\section{Project presentation \& User experience}
- How the interaction loop works? What the user sees and how the application behaves with respect to his/her input.

- Relevant screenshot(s) (and/or photos showing the user playing) of the program running

%------------------------------------------------

\section{Libraries, resources and algorithms used}
The virtual environment was built using Unity.

The project uses the Oculus integration plugin for displaying the VE and to achieve head tracking. A modified version of the Hydra framework provided by Razer is used for the tracking of the markers. Full use is made of the various C\# libraries provided by Unity including utilities, math, etc.

The physic simulation and particle systems generation mechanisms are those of Unity.

%------------------------------------------------
	
	\section{Understand and compile the project}
	Since the project uses the Sixsense Razor Hydra as a marker tracking system, whose driver is only available for Windows, it is only compatible with a Windows PC. The Oculus and Hydra drivers must be installed on the computer. The version of the different components are the following:
	\begin{description}
		\item[Oculus VR Unity integration] v0.5.0.1-beta
		\item[Sixsense Unity plugin] v1.0.6
		\item[Oculus driver] Windows - v0.5.0.1-beta
		\item[Razer Hydra driver] Windows - v1.01
	\end{description}
	
	
	The whole project can be opened in Unity and compiled from there. The most relevant C\# files to look at in order to understand the project are:
	\begin{description}
		\item[\texttt{planetController.cs}] Generate planets and star clusters according to a given filesystem path. It is also responsible for the solar system movement and the file display 
		\item[\texttt{UserController.cs}] Controls player movements. Contains methods called by the Sixsense scripts.
		\item[\texttt{SixenseHand}] Animate the hand and detects controller movement corresponding to action.
		\item[\texttt{SicenseHandsController}] Handle some over movement that need both hands and correspond to action.
	\end{description}
	
	Once compiled, the \texttt{FileExplorerDirectToRift.exe} file can then be launched to run the file explorer.
	
	Maybe include a diagram and code snippets?
	
	%------------------------------------------------

\section{Discussion \& Conclusion}

During this project we learned a lot of thing regarding the field of VR itself and more generally concerning this kind of creative projects. Given the time this project was supposed to take we had to leave aside lots of improvements we discussed along the path, such as being able to launch executable applications, display images and videos, and so on. On the other hand we believe our code is ready to house these features since we made it in a clear and modulated fashion.

What was good about this project was not to have to stumble upon OpenGL issues and really being able to focus on the VR side of the project. On the other hand it was unclear how much time, effort and quality was expected, which has led us to quite a large workload.

%------------------------------------------------

\section{References}
\label{sec:refs}
\begin{enumerate}
\item ref 1: \url{https://www.youtube.com/watch?v=jLp3W1gbhRk&t=51}
\end{enumerate}

%----------------------------------------------------------------------------------------

\end{document}
